%iffalse
\let\negmedspace\undefined
\let\negthickspace\undefined
\documentclass[journal,12pt,twocolumn]{IEEEtran}
\usepackage{cite}
\usepackage{amsmath,amssymb,amsfonts,amsthm}
\usepackage{algorithmic}
\usepackage{graphicx}
\usepackage{textcomp}
\usepackage{xcolor}
\usepackage{txfonts}
\usepackage{listings}
\usepackage{enumitem}
\usepackage{mathtools}
\usepackage{gensymb}
\usepackage{comment}
\usepackage[breaklinks=true]{hyperref}
\usepackage{tkz-euclide} 
\usepackage{listings}
\usepackage{gvv}                                        
\def\inputGnumericTable{}                                 
\usepackage[latin1]{inputenc}                                
\usepackage{color}                                            
\usepackage{array}                                            
\usepackage{longtable}                                       
\usepackage{calc}                                             
\usepackage{multirow}                                         
\usepackage{hhline}                                           
\usepackage{ifthen}                                           
\usepackage{lscape}
\usepackage{tabularx}
\usepackage{array}
\usepackage{float}


\newtheorem{theorem}{Theorem}[section]
\newtheorem{problem}{Problem}
\newtheorem{proposition}{Proposition}[section]
\newtheorem{lemma}{Lemma}[section]
\newtheorem{corollary}[theorem]{Corollary}
\newtheorem{example}{Example}[section]
\newtheorem{definition}[problem]{Definition}
\newcommand{\BEQA}{\begin{eqnarray}}
\newcommand{\EEQA}{\end{eqnarray}}
\newcommand{\define}{\stackrel{\triangle}{=}}
\theoremstyle{remark}
\newtheorem{rem}{Remark}

% Marks the beginning of the document
\begin{document}
\bibliographystyle{IEEEtran}
\vspace{3cm}

\title{Title of your Document}
\author{EE24BTECH11045-Namala Tapasvi$^{*}$%}
\maketitle
\newpage
\bigskip

\renewcommand{\thefigure}{\theenumi}
\renewcommand{\thetable}{\theenumi}\documentclass{article}
\usepackage{multicol}
\pagestyle{empty}
\usepackage{graphicx} % Required for inserting images
\usepackage{amsmath}
\usepackage{enumitem}
\usepackage{esami}

\title{Assignment 1}
\author{Namala Tapasvi}
\date{August 10,2024}

\begin{document}

\maketitle{CHAPTER 11 Limits,Continuity and Differentiability}\\

Section-A JEE Advanced/IIT JEE\\

\begin{multicols}{2}

\section*{A  Fill in the Blanks}
\begin{enumerate}
    \item  Let $f(x)=\{(x-1)^2\sin\frac{1}{(x-1)}-|x|  if x\neq1 \\ -1 if x=1\}$\\ be a real-valued function. Then the set of points where f(x) is not differentiable is.........\hfill (1981-2marks)\\ 

    \item Let $f(x)=\{\frac{(x^3+x^2-16x+20)}{(x-2)^2}, if x\neq2\\ k, if x=2\}$\\If f(x) is continuous for all x,then k=........\hfill (1981-2 marks)\\

    \item A discontinuous function y=f(x) satisfying $x^2+y^2=4$ is given by f(x)=.........\hfill (1982-2 marks)\\

    \item $\lim \limits_{x \to -\infty}(1-x)\tan\frac{\pi x}{2}=........$\hfill (1982-2marks)\\

    \item If $f(x)=\sin x,x\neq n\pi,n=0,\pm 1,\pm 2,\pm 3\\=2 otherwise \\ and g(x)=x^2+1,x\neq0,2\\=4,x=0\\=5,x=2\\ then \lim \limits_{x \to 0}g[f(x)] is........ \hfill (1986-2marks)$\\

    \item $\lim \limits_{x \to -\infty} [\frac{x^4 \sin (\frac{1}{x})+x^2}{(1+|x|^3)}]=.......$\hfill     (1987-2marks)\\

    \item If f(9)=9, f'(9)=4 $\lim \limits_{x \to 9} \frac{\sqrt{f(x)}-3}{\sqrt{x}-3}$ equals..........\hfill (1988-2marks)\\

    \item ABC is an isosceles triangle inscribed in a circle of radius r. If AB=AC and h is altitude from A to BC then the triangle has perimeter ${P=2(\sqrt{2hr-h^2})+\sqrt{2hr}}$ and A is ........ also $\lim \limits_{h \to 0} \frac{A}{P^3}=..........\hfill (1982-2marks)$\\

    \item $L\text{t}_{x \to \infty} (\frac{x+6}{x+1})^(x+4)=...........$ \hfill (1990-2marks)\\

    \item Let $f(x)=x|x|$. The set of all points where f(x) is twice differentiable is.......\hfill (1992-2marks)\\

    \item Let $f(x)=[x] \sin\frac{\pi}{[x+1]}$, where [.] denotes greatest integer function. The domain of f is...... and the points of discontinuity of f in domain are.......\hfill (1996-2marks)\\

    \item $\lim \limits_{x \to 0} (\frac{1+5x^2}{1+3x^2})^\frac{1}{x^2}=............\hfill (1996-2marks)$\\

    \item Let f(x) be a continuous function defined for $1\leq x \leq 3$. If f(x) takes rational values for all x and f(2)=10 then f(1.5)=.........\hfill (1997-2marks)\\
\end{enumerate}
\end{multicols}

\section*{B True/False}
\begin{enumerate}
    \item $L\text{t}_{x \to a} [f(x)g(x)]$ exists then both $L\text{t}_{x \to a} f(x)$ and $L\text{t}_{x \to a} g(x)$ exist.\hfill (1981-2marks)\\
\end{enumerate}


\section*{C MCQs with One Correct Answer}
\begin{enumerate}
\item If $f(x)=\sqrt{\frac{x-\sin x}{x+\cos^2 x}}$ \hfill (1979)
\begin{enumerate}
\begin{multicols}{2}
    \item 0
    \item 1
    \item $\infty$
    \item none of these
\end{multicols}
\end{enumerate}
\end{enumerate}
\end{document}
