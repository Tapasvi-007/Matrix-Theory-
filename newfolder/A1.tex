\let\negmedspace\undefined
\let\negthickspace\undefined
\documentclass[journal]{IEEEtran}
\usepackage[a5paper, margin=10mm, onecolumn]{geometry}
\usepackage{lmodern} % Ensure lmodern is loaded for pdflatex
\usepackage{tfrupee} % Include tfrupee package
\usepackage[utf8]{inputenc}
\usepackage[T1]{fontenc}

\setlength{\headheight}{1cm} % Set the height of the header box
\setlength{\headsep}{0mm}     % Set the distance between the header box and the top of the text

\usepackage{gvv-book}
\usepackage{gvv}
\usepackage{cite}
\usepackage{amsmath,amssymb,amsfonts,amsthm}
\usepackage{algorithmic}
\usepackage{graphicx}
\usepackage{textcomp}
\usepackage{xcolor}
\usepackage{txfonts}
\usepackage{listings}
\usepackage{enumitem}
\usepackage{mathtools}
\usepackage{gensymb}
\usepackage{comment}
\usepackage[breaklinks=true]{hyperref}
\usepackage{tkz-euclide} 
\usepackage{listings}
\usepackage{gvv}                                        
\def\inputGnumericTable{}                                 
%\usepackage[latin1]{inputenc}
\usepackage{color}                                            
\usepackage{array}                                            
\usepackage{longtable}                                       
\usepackage{calc}                                             
\usepackage{multirow}                                         
\usepackage{hhline}                                           
\usepackage{ifthen}                                           
\usepackage{lscape}

\begin{document}
\bibliographystyle{IEEEtran}
\vspace{3cm}

\title{1.1.5.15}
\author{EE24BTECH11045 - N.Tapasvi}
{\let\newpage\relax\maketitle}
Question:\\
The midpoint of the line segment joining $\vec{A}\myvec{2a\\4}$ and $\vec{B}\myvec{\text{-}2\\3b}$ is $\vec{M}\myvec{1\\2a+1}$. Find the values of a and b.
\hfill (10,2019)

\solution
\begin{table}[h!]    
  \centering
  \begin{tabular}[12pt]{ |c| c|}
    \hline
    \textbf{Variable} & \textbf{Description}\\ 
    \hline
	$\vec{A}$ & $\myvec{2a\\4}$\\
	\hline
	$\vec{B}$ & $\myvec{-2\\3b}$\\
	\hline
	$\vec{C(Midpoint)}$ & $\myvec{1\\2a+1}$\\
	\hline
	$\vec{a,b}$ & Values to be found\\
\end{tabular}

  \caption{Variables Used}
  \label{tab1-1.5-15}
\end{table}\\

Let M divide AB in the ratio k:1
then,M=\[\frac{kB+A}{k+1}\]\\

%M=\[\frac{\myvec{\text{-}2k+2a\\3bk+4}}{k+1}\]\\

As M is the midpoint k=1\\

%M=\[\frac{\myvec{\text{-}2+2a\\3b+4}}{2}\\
%\quad \Rightarrow \quad
%\myvec{1\\2a+1}=\frac{\myvec{\text{-}2k+2a\\3bk+4}}\]
%\quad \Rightarrow \quad a=2,b=2
Let the midpoint \( \vec{M} \) be given by the formula:
\[
\vec{M} = \frac{\vec{A} + \vec{B}}{2}
\]
Substituting the coordinates of \( \vec{A} \) and \( \vec{B} \):
\[
\vec{M} = \frac{1}{2} \begin{pmatrix} 2a - 2 \\ 4 + 3b \end{pmatrix}
\]
Since \( \vec{M} = \begin{pmatrix} 1 \\ 2a+1 \end{pmatrix} \), we equate the corresponding components:
\[
\frac{2a - 2}{2} = 1 \quad \text{and} \quad \frac{4 + 3b}{2} = 2a + 1
\]
From the first equation:
\[
2a - 2 = 2 \quad \Rightarrow \quad 2a = 4 \quad \Rightarrow \quad a = 2
\]
Substitute \( a = 2 \) into the second equation:
\[
\frac{4 + 3b}{2} = 2(2) + 1 = 5 \quad \Rightarrow \quad 4 + 3b = 10 \quad \Rightarrow \quad 3b = 6 \quad \Rightarrow \quad b = 2
\]

Thus, \( a = 2 \) and \( b = 2 \).


\begin{figure}[h!]
   \centering
	\includegraphics[width=1.15\linewidth]{/home/namala-tapasvi/fig.png}
   \caption{Plot of the points A,B,M}
   \label{stemplot}
\end{figure}

\end{document}

